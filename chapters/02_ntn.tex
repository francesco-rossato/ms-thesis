%!TEX root = ../main.tex

\chapter{Non-terrestrial networks}
\label{chp:ntn}

\section{Satellite types}
\label{sec:satellite-types}
Satellites are divided in three main different categories depending on their orbiting altitude: \ac{GEO}, \ac{MEO}, and \ac{LEO} satellites. Each one has its own characteristics, presenting some upsides and downsides as described below. Fig. \ref{fig:satellite_coverages} illustrates the different orbiting altitudes as well as an approximate idea of the different coverage areas for each orbit.

\begin{figure}[ht]
    \centering
    \includegraphics[width=0.9\textwidth]{res/satellite-coverages.jpg}
    \caption{Height and approximate coverage areas for satellites at different orbits \cite{sustainable-sat-com-6g}}
    \label{fig:satellite_coverages}
\end{figure}

\subsection{GEO satellites}
Orbiting at a height of 35.786Km, to an observer placed on the Earth surface \ac{GEO} satellites appear stationary, since their orbiting period is the same as the Earth rotational period.

\paragraph{Advantages}    
Since \ac{GEO} satellites are geostationary, continuous coverage to a designated area can be provided using as little as a single satellite, while the use of non-\ac{GEO} satellites would require the deployment of a constellation, which is both more complex and more expensive. 

This also vastly simplifies the problem of the ground equipment having to be able to track the satellite. Since the satellite position is always known, once the position of the \ac{UE} is established, the relative position of the satellite can be easily calculated.

As shown in Fig. \ref{fig:satellite_coverages}, the high altitude of \ac{GEO} satellites creates a large cell footprint. While the deployment cost of a single \ac{GEO} satellite is higher than both \ac{MEO} and \ac{LEO} ones, the cost per coverage area is overall lower, and an almost full coverage of the terrestrial globe can be achieved using only three equally spaced satellites \cite{types-of-orbits-esa}.

\paragraph{Disadvantages}
The disadvantages of \ac{GEO} satellites are mainly linked to the large distance with \ac{UE}s places on the Earth surface: the transmission power and the antenna gain have to be high enough to overcome the greater propagation losses, and the propagation delay of the signal adds about 120ms to the overall latency. This means that if the \ac{UE} sends a request to a server at $t=0$ through a \ac{GEO} link, the packet will be received by the destination node at least at $t=240$ms. The response will then finally reach the \ac{UE} after at least $480$ms from the initial transmission, and these calculations do not factor in any delay related to medium access requests, packet transmission times and processing delays, which would further increased the overall latency.

In addition to the positive aspects previously discussed, the large cell footprint also brings some downsides with it. Due to the vast area, a single satellite will be required to serve a massive number of users, so the total available capacity will have to be shared between a bigger number of equipments, and the throughput experienced by each of them will be reduced.

Solutions to provide a greater capacity have been proposed and are currently in the early stage of studies, such as the use of beamforming to divide the covered area in smaller cells and the employment of higher frequency bands towards Ku, K and Ka as depicted in Fig. \ref{fig:satellite-bands} \cite{advances-comm-sat-sys}.
Moreover, the high number of users also leads to a large rate of initial access requests, with the possibility of channel saturation as described in \cite{3gpp-tr-38.811}.

\begin{figure}[ht]
    \centering
    \includegraphics[width=0.9\textwidth]{res/satellite-bands.png}
    \caption{Satellite spectrum bands allocation \cite{advances-comm-sat-sys}}
    \label{fig:satellite-bands}
\end{figure}

\subsection{MEO satellites}
\ac{MEO} orbit is comprised between \ac{LEO} and \ac{GEO}, therefore all satellites orbiting between 2000Km and 35.786Km are considered as \ac{MEO}. This vast orbital space is mainly used by navigation systems such as GALILEO \cite{types-of-orbits-esa}.

The propagation delay can vary a lot depending on the altitude, but it is larger than \ac{LEO} and smaller than \ac{GEO}. The same point can be made regarding the cell size and number of served users. 

As for \ac{LEO} satellites, \ac{MEO} ones do require a constellation in order to provide continuous coverage over a designated area, since they are not geostationary. The numerology of the constellation, however, is smaller than the one required for \ac{LEO} satellites, since each platform can serve a larger area.

Their peculiarity of presenting many of the downsides that characterize both \ac{GEO} and \ac{LEO} satellites, while being unable to offer any substantial benefit over their competitors besides the need for a smaller constellation, makes them less than ideal candidates for applications in non-terrestrial networks.

\subsection{LEO satellites}
\label{sec:leo}
Orbiting below the threshold altitude of 2.000Km, \ac{LEO} satellites are the most promising solution in the realm of \ac{NTNs} mainly because they can offer really advantageous throughput and propagation delay. The disadvantages and complications are however many, but their benefits overshadow them.

\paragraph{Advantages}
The lower altitude entails a shorter propagation delay, on the order of about 6ms, and the smaller coverage area of each satellite depicted in Fig. \ref{fig:satellite_coverages} means that the total number of users that need to be served is smaller. This also allows the use of higher frequency bands and the constraints of high antenna gains are less stringent compared to \ac{GEO} satellites, since the experienced path loss is much smaller. This in turn enables the achievement of higher overall throughputs, more suited to satisfy the requirements of modern days broadband connectivity, as detailed in \cite{satellite-communication-mmwave-giordani}.

While not being numerous, those advantages are enough to make \ac{LEO} satellites the most promising choice in the field of \ac{NTN}s.

\paragraph{Disadvantages}
The cost per deployed satellite is significantly smaller than \ac{GEO} and \ac{MEO} satellites, and multiple deployments within a single launch are possible, driving the costs further down. However, given that \ac{LEO} satellites are not geostationary, a large constellation is needed to provide a continuous service, driving the deployment costs significantly up. As an example of how vast those constellations can become, Fig. \ref{fig:starlink_constellation} depicts the \ac{LEO} satellites employed by Starlink\footnote{Starlink is a satellite internet constellation operated by Starlink Services, LLC, a wholly-owned subsidiary of American aerospace company SpaceX.}, counting about 4.808 units in service at the time of writing\footnote{Source: \href{https://satellitemap.space/}{\texttt{satellitemap.space}}}.

\begin{figure}[ht]
    \centering
    \includegraphics[width=0.9\textwidth]{res/starlink-constellation.png}
    \caption{Starlink constellation as July 2024. Source: \href{https://satellitemap.space/}{\texttt{satellitemap.space}}}
    \label{fig:starlink_constellation}
\end{figure}

Since low orbiting satellites remain in view of the user equipments only for a short period of time, with an average in-view duration of just 13 minutes as calculated in \cite{regional-coverage-analysis-leo}, all the connected users are expected to be handed over to the next available satellite within this time window. Such behavior would create a noticeable protocol overhead, consuming available channel capacity and potentially adding more latency. However, the predictable nature of this phenomenon might allow for a partial automation without requiring data to be exchanged.

The small coverage area means that more terrestrial gateways have to be deployed, since each satellite can only communicate with the ground via the terrestrial gateways that fall within its view. A different solution to the densification of gateways is the use of \ac{ISL}: high-bandwidth links between different satellites of the constellation, capable of connecting satellites that do not have gateways in sight to ones that are connected to a gateway, allowing traffic to be routed to the ground with additional hops. Inter-satellite links have also to be implemented if coverage over the oceans is required. This ultimately adds up to the already high constellation deployment costs.

A problem affecting all the non-geostationary satellites involves their speed relative to the user equipment located on the ground. A \ac{LEO} satellite moves with a speed of about 7.8Km/s \cite{leo-definition-theory-facts}, presenting therefore a noticeable Doppler shift. This has to be compensated for, and preliminary solutions in this sense require usage of \ac{UE}s with GNSS capabilities, which is not always a reasonable assumption \cite{satellite-communication-mmwave-giordani, 3gpp-tr-38.821}. Moreover, the dependency of the network on a third-party service in order to operate properly would add a point of failure outside the control of the network operators, and a disruption in GPS service would halt the network capabilities.

\subsection{Multilayered networks}
The aforementioned solutions are not to be considered mutually exclusive. In fact, the multitude of possibilities that their combinations offer opens to the study of various different scenarios, in which the upsides of the space, air, and ground layers are orchestrated to improve quality of service. Fig. \ref{fig:multilayered-ntn} showcases a highly sophisticated non-terrestrial multilayered network scenario where different access technologies are used.

\begin{figure}[ht]
    \centering
    \includegraphics[width=0.9\textwidth]{res/multilayered-ntn.jpg}
    \caption{Complex multilayered \ac{NTN} scenario \cite{connecting-ntn-rohde-schwarz}}
    \label{fig:multilayered-ntn}
\end{figure}

In \cite{potential-multilayered-nierarchical-ntn-wang} it has been shown that the use of \ac{HAP} as relays between the ground segment of the network and the upper \ac{GEO} satellite links can deliver up to six times the capacity, and better overall outage probability, than point-to-point \ac{GEO} transmissions.


\section{Types of payloads}
When implementing a non-terrestrial network, an important choice to be made is the type of payload to use. Each one of the two main presented categories has its own benefits as briefly described below.

\subsection{Bent-pipe payload}
\label{sec:bent-pipe-payload}
This is the simplest approach, where the role of the satellite consists only of  repeating the signal received from the \ac{UE} on the ground towards the terrestrial gateway. Configurations such as the one depicted in Fig. \ref{fig:ntn-bent-pipe} go by the name of bent pipe payloads and are characterized by the presence of a terrestrial g-NodeB, while the satellite has the sole purpose of providing a transparent link with the \ac{UE}.

While such solutions are by far the simplest in terms of payload complexity, the main drawback is the even longer experienced latency, since communications between users served by the same satellite would also have to be routed through the terrestrial gateway, increasing the latency by at least two times the propagation delay.
This scenario also poses strict bandwidth requirements on the feeder link, since all the traffic must necessarily pass through it.
Other, smarter, solutions are able to route at least part of the inbound traffic autonomously, without routing everything back to earth.

\begin{figure}[ht]
    \centering
    \includegraphics[width=0.9\textwidth]{res/ntn-bent-pipe.png}
    \caption{NTN with access network based on satellites with bent pipe payload \cite{3gpp-tr-38.811}}
    \label{fig:ntn-bent-pipe}
\end{figure}

\subsection{On-board g-NodeB}
\label{sec:onboard-gnb}
A slightly more sophisticated approach foresees the installation of g-NodeB capabilities directly onto the satellite payload. This has the benefit of reducing the experienced latency in some cases, as well as reducing the utilization of the feeder link. Certain protocols, designed to terminate at the \ac{gNB}, can in this case reach their designated endpoint without necessitating to be routed back to the ground gateway.

Figure \ref{fig:ntn-gnb-onboard} from \cite{3gpp-tr-38.811} shows the architecture of a \ac{NTN} using access network based on satellites with \ac{gNB} on board as payload.

\begin{figure}[ht]
    \centering
    \includegraphics[width=0.9\textwidth]{res/ntn-regen.png}
    \caption{NTN with access network based on satellites with on-board gNB payload \cite{3gpp-tr-38.811}}
    \label{fig:ntn-gnb-onboard}
\end{figure}

\section{Available commercial solutions}
\paragraph{Legacy solutions}
Commercial solutions, first concerning satellite-based phone calls and successively evolved to provide internet access, have been available since a long time. However, the majority of this legacy infrastructure makes use of \ac{GEO} satellites, therefore presenting all the limitations discussed in section \ref{sec:satellite-types}, offering limited throughput and large delays. Such constraints render this technology not suitable for the needs of modern internet connections standards.

\paragraph{Recent developments}
While commercial solutions using \ac{LEO} satellites are starting to appear, enjoying some degree of success with notable examples such as the one presented in section \ref{sec:leo}, all of those solutions make use of many proprietary protocols to allow for the communication to take place, and up until now no internationally agreed standard has been defined.

This is where most of the research is currently being conducted and where the business world is posing its attention due to the characteristics of \ac{LEO} satellites that makes them the most suited to provide high speed satellite internet access.