%!TEX root = ../main.tex

\chapter{State of the art}
\label{chp:state_of_the_art}

This chapter is focused on the discussion of the simulation software used to 
\section{Simulator}
The results presented in this work have been obtained using the ns-3 network simulator tool. Ns-3 is a discrete-events network simulator specifically targeted to the research world.

There are a few reasons why this choice was made. First and foremost, the fact that this is a full-stack simulator played a crucial role, 

\section{Scenario}
The simulated scenario consisted of a simple setup including a single \ac{UE} ground terminal connected to a satellite acting as \ac{gNB}. The satellite is then connected via an ideal link with no latency and high throughput to a node that acts as second endpoint of the communication.

The parameters of the satellite antenna follow the scenario 10 DL indicated in TR 38.321 \todo{insert reference} and are reported in the code snippets \ref{code:sat_parameters} and \ref{code:ue_parameters}.

\begin{lstlisting}[language=C++, caption=Satellite antenna parameters, label=code:sat_parameters]
    // Satellite parameters
    double satEIRPDensity = 40;    // dBW/MHz
    double satAntennaGain = 58.5;  // dB
    double satAntennaDiameter = 5; // meters
\end{lstlisting}

\begin{lstlisting}[language=C++, caption=UE antenna parameters, label=code:ue_parameters]
    // UE Parameters
    double vsatAntennaGain = 39.7;       // dB
    double vsatAntennaDiameter = 0.6;    // meters
    double vsatAntennaNoiseFigure = 1.2; // dB 
\end{lstlisting}


\paragraph{}\todo{move this to state of the art} Focusing on the \ac{MAC} sublayer, the large propagation delay of satellite links affects different aspects, making the actual implementation not suited for a \ac{NTN} scenario. In the \ac{HARQ} protocol, the retransmission timeout is likely to expire before a single \ac{RTT}, leading to unnecessary retransmissions. Moreover, the limit on the maximum number of concurrent \ac{HARQ} processes leads to a stop-and-wait behaviour, which may increase the energy consumption \cite{3gpp-tr-38.811}. On the other hand, it has been noted that disabling \ac{HARQ} would lead to an even worse performance penalty, therefore requiring a redesign for \ac{NTN} \cite{5g-beyond-5g-ntn-trends-vanellicoralli}. Another 5G \ac{NR} protocol which is negatively impacted in \ac{NTN} is the initial access, since users at the center of the cell face a smaller propagation delay with respect to users at the cell edge \cite{5g-beyond-5g-ntn-trends-vanellicoralli} \cite{applying-nr-technologies-in-ntn-lee}. As a result, preambles of \ac{UE}s placed near the cell edge may reach the satellite when the \ac{RACH} opportunity has already expired, which may lead to collisions. During the initial access phase, \ac{UE}s are not aware of their propagation delay, and the high mobility of \ac{gNB}s on \ac{LEO} satellites causes a non-negligible Doppler shift. Those factors vary with the relative position and speed between the \ac{UE} and the \ac{gNB}, and the protocols for initial access must be modified in \ac{NTN} to account for them \cite{ntn-from-5g-6g-hassan}. 
