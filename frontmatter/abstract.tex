%!TEX root = ../main.tex
\begin{abstract}
    The advent of 5G technology has revolutionized the world of mobile communication, offering higher throughput, lower latency and high reliability. As the demand for connectivity continues to grow, non-terrestrial networks (NTNs) have been identified by the Third generation Partnership Project as crucial for future 5-6G deployments, complementing terrestrial networks with the aim of increasing resiliency and providing connectivity in remote areas.
    The characteristics of NTNs, such as the long distances involved, the high propagation delay and the larger cell footprint, introduce complexities not typically encountered in their terrestrial counterpart, and the current protocol stack was not designed with such challenges in mind.
    Through a comprehensive literature review and the use of ns-3 network simulator, this work identifies key characteristics and constraints of non-terrestrial networks, evaluating existing protocols and highlighting their strengths and limitations in this challenging scenario, before finally delving into the adaptation and optimization of 5G New Radio Layer-2 protocols for NTNs, proposing innovative solutions and paving the way for new applications and services that leverage the full potential of 5G technology beyond traditional terrestrial boundaries.
\end{abstract}