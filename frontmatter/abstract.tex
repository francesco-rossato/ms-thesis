%!TEX root = ../main.tex
\begin{abstract}
    The advent of 5G technology has revolutionized the world of mobile communication, offering high throughput, low latency and high reliability. As the demand for connectivity continues to grow, non-terrestrial networks (NTNs) have been identified as crucial solutions for future 5-6G deployments, complementing terrestrial networks with the aim of increasing resiliency and providing connectivity in remote areas.
    The characteristics of NTNs, such as the long distances involved, the high propagation delay and the larger cell footprint, introduce new and more critical complexities than in terrestrial networks, which require a thorough re-definition and re-design of most of the protocol stack.

    After a careful literature review, in this thesis we identify key characteristics and constraints of NTNs, highlighting strengths and limitations of legacy 3GPP NR protocols in the NTN scenario. In light of this, we propose and implement new 5G NR Layer-2 protocols specifically tailored to NTN, focusing in particular on retransmission management via Hybrid Automatic Repeat reQuest and scheduling. The proposed solutions have been implemented in the ns-3 end-to-end full-stack simulator by extending the ns3-ntn module, which guarantees the realism and accuracy of the results.

    This work resulted in a major improvement of the ns-3 ntn module, whose implementation of the scheduler is now capable of tolerating scenarios with high propagation delays typical of NTNs. The numerical results obtained from the simulations show a measurable improvement in terms of reliability and throughput. Problems such as expiring timers, inflated buffer status reports, and unnecessary retransmissions were documented and addressed.

    We then focused the attention to the implementation of the HARQ protocol, that initially caused the throughput to drop significantly. We argued that its use is still beneficial in contexts of low SNR, while implementing modificaitons that increased the maximum achievable throughput.

\end{abstract}