\begin{abstract}[it]

    Through a comprehensive literature review and the use of ns-3 network simulator, this work identifies key characteristics and constraints of non-terrestrial networks, evaluating existing protocols and highlighting their strengths and limitations in this challenging scenario, before finally delving into the adaptation and optimization of 5G New Radio Layer-2 protocols for NTNs, proposing innovative solutions and paving the way for new applications and services that leverage the full potential of 5G technology beyond traditional terrestrial boundaries.

    L'avvento della tecnologia 5G ha rivoluzionato il mondo delle comunicazioni cellulari, offrendo througput più elevati, minor latenza ed elevata affidabilità. Il crescere della domanda di connettività di rete ha portato il Third Generation Partnership Project (3GPP) a indicare le reti di comunicazione non terrestri (NTNs) come cruciali per le future reti di 5-6G. Il loro scopo è di complementare le reti di comunicazione terrestri, aumentandone la resilienza e fornenedo connettività nelle aree più remote.
    Le caratteristiche delle NTNs, come le lunghe distanze tra gli apparati a terra e la componente satellitare, i lunghi ritardi di propagazione e le grandi dimensioni delle celle, introducono delle complessità nuove rispetto alle reti terrestri, e lo stack protocollare attualmente in uso non è disegnato per affrontare queste sfide.



\end{abstract}